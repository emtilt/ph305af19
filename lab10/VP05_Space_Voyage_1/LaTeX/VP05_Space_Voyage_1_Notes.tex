\documentclass[11pt]{article}
\usepackage{tabularx}
\usepackage[margin=1in]{geometry}
\usepackage{fancyhdr} 
\pagestyle{empty}

\begin{document}
\begin{noindent}
\begin{flushleft}

%New Command
\newcommand{\detail}[2]
{\begin{tabularx}{6.5in}{p{2in} X}
\textbf{#1:} & #2
\end{tabularx}
\vspace{11pt}}

%Activity
\detail{Name of Activity}{A spacecraft voyage, part 1: Spacecraft and Earth }\\
\detail{Approximate Duration}{45 minutes}\\
\detail{Relevant Section of Text}{3.6}\\

\begin{description}
\item[ Write on Board] 
\end{description}

\begin{itemize}
\item Review previous programs (Motion part 2 \& Gravitational Force Calculation).
\item Read and Follow Instructions!
\item \textbf{Do not run the program} until directed to do so in the instructions.
\end{itemize}

\begin{description}
\item[Checking Questions/Things to look for] 
\end{description}

\begin{itemize}
\item Students must make a prediction of what they would expect to see when they run the program before doing so.
\item If students produce an incorrect prediction, ask students to justify why their prediction was incorrect, or why the visual output produced the observed behavior of the spacecraft.
\item Encourage groups to use earlier programs for reference.
\item Emphasize the importance of initial conditions and time steps.
\item Make sure they only update the position and axis of the arrow inside the while loop and NOT the whole arrow.
\item Ask them the questions from the handout and emphasize the difference between momentum and change in momentum.
\begin{itemize}
\item Why does spacecraft move faster when closer to the Earth?
\subitem \textbf{Main Points:}
\begin{itemize}
\item \textbf{NOT} because the force is bigger (this implies the mistaken idea that $p$ is proportional to $F$ rather than $\Delta p$ is proportional to $F$).
\item On the way toward the Earth there is a parallel component of the force in the direction of motion, which increases the magnitude of momentum.
\item It goes slower away from the Earth because on the way away from the Earth there is a parallel component of the force opposite to the motion, which decreases the magnitude of momentum.
\item Use example of spring-mass system to make the contrast: the net force is greatest when the speed is zero, and the net force is smallest when the mass has its maximum speed.
\end{itemize}
\item What is the direction of momentum?
\subitem \textbf{Main Point:} Along the direction of motion
\subitem ($\Delta \vec{p}$ is in the direction of the net force)
\end{itemize}
\item Final program should have an elliptical orbit and an arrow to represent the momentum.
\end{itemize}

\begin{description}
\item[Grading] 
\end{description}

\begin{itemize}
\item Program is running and looks reasonable
\item Proper initial conditions
\item Correct physics
\begin{itemize}
\item Gravitational force calculated correctly
\item Position update and momentum principle used correctly
\item Arrow points in the correct direction
\item Arrow has correct magnitude (should represent momentum) and changes appropriately
\end{itemize}
\item Reasonable scale factor
\item Extra credit for force arrow
\end{itemize}

\begin{description}
\item[General Grading Key for VPython] 
\end{description}
\begin{itemize}
\item 100\% of total points for a program that runs correctly and exhibits all the required features.
\item 0\% for a program that does not run (i.e. produces an error message and quits).
\item For partially correct or partially complete programs, give:
\begin{itemize}
\item 90\% for a program with no physics errors, but one missing feature
\item 80\% for a program with no physics errors, but several missing features
\item 50-70\% for a program with some physics errors
\item 40\% for a rudimentary program that is not complete
\end{itemize}
\end{itemize}

\end{flushleft}
\end{noindent}
\end{document}