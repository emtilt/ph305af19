\documentclass[12pt]{article}

\newcommand{\squishlist}{
   \begin{list}{$\bullet$}
    { \setlength{\itemsep}{0pt}      \setlength{\parsep}{3pt}
      \setlength{\topsep}{3pt}       \setlength{\partopsep}{0pt}
      \setlength{\leftmargin}{1.5em} \setlength{\labelwidth}{1em}
      \setlength{\labelsep}{0.5em} } }

\newcommand{\squishend}{
    \end{list}  }

\usepackage{times}
\usepackage{mathptm}
\usepackage{fullpage}
\usepackage{graphicx}

\begin{document}

\begin{center}
{\bf{Regis University -- Physics 305A -- Fall 2019}} \\
{\bf{Lab 8: Energy in Springs}} \\
\medskip
\end{center}

\medskip

In this lab, we will check our understanding of transformations of 
energy. Specifically, we will work with kinetic 
energy ($K = \frac{1}{2} m v^2$), gravitational potential 
energy ($\Delta U_g = m g \Delta y$), and potential energy stored in a 
spring ($U_s = \frac{1}{2} k_s s^2$).  This lab handout is 
intentionally a short outline; you are expected to fill in the 
details of the procedure based on your understanding of the physical
principles. This week you're welcome to use whatever calculational tools you prefer -- Excel, Google Sheets, Python, etc.

\bigskip

\noindent {\bf {Part 1 -- Measuring spring stiffness}}

\medskip

There is a general connection between potential energy and force,
expressed by $F_x = - \frac{d U} {d x}$.  \\
If we apply this to a spring's potential energy, we find that 
$F_s = - \frac{d}{d s} (\frac{1}{2} k_s s^2) = - k_s s$, a relationship
that is often called ``Hooke's Law.''  It indicates that the force
exerted by a spring is proportional to the distance by which the spring
has been stretched.
You should be able to use this relationship to set up an experiment 
to measure the spring stiffness (often called the ``spring constant'') 
$k_s$ for a particular spring by 
hanging different masses from the spring and measuring by how much it
stretches.  Specifically, you should determine it from a {\bf graph} of 
$F_s$ versus $s$; what do you expect the slope and the intercept of 
that graph to represent?

In Logger Pro, you can enable an option to display its calculation of 
the uncertainties on the slope and intercept of a graph.  You should collect at 
least enough data to determine $k_s$ with an uncertainty of less than about
5 percent of the value.

Use this method to determine the stiffness of two different springs, 
and compare them to their nominal values.  Does your measured stiffness
agree within the calculated uncertainty?  Does the intercept of your graph 
agree with what you expect, to within its uncertainty?

\bigskip

\noindent {\bf {Part 2 -- Conservation of energy}}
\medskip

You can hang a known mass from one of the springs that you have now 
characterized, and you can allow it to bounce up and down.  This system
is called a ``simple harmonic oscillator.''  Set up your oscillator so that 
it bounces above an ultrasonic motion detector, which should show
you beautiful sinusoidal graphs of the vertical position $y$ and the
vertical velocity component $v_y$ versus time.
Determine values for $K$, $U_g$, and $U_s$ at three points in the motion of 
the mass:
\squishlist
\item the highest point
\item the center of the motion
\item the lowest point
\squishend
To what extent does the total energy $K + U_g + U_s$ remain constant
over this full cycle?

\end{document}
