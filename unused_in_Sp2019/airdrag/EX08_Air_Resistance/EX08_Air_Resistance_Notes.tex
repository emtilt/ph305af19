\documentclass[a4paper]{article}
\usepackage{tabularx}
\usepackage{amsmath}
\usepackage{amssymb}

\begin{document}
\begin{noindent}
\begin{flushleft}

%New Command
\newcommand{\detail}[2]
{\begin{tabularx}{12 cm}{p{5cm} X}
\textbf{#1:} & #2
\end{tabularx}
\vspace{0.5cm}}

%Activity
\detail{Name of Activity}{The Air Resistance Force}\\
\detail{Approximate Duration}{40 minutes}\\
\detail{Relevant Section(s) of Text}{7.10}\\

\begin{description}
\item[ Write on Board] 
\end{description}

\begin{itemize}
\item Read entire handout \textbf{before} beginning
\item Drop filters from near the ceiling
\item Use Excel to graph $F_{\text{air}}$ vs. terminal speed
\item Use ``power'' fit for trendline (NOT ``linear'' fit)
\end{itemize}

\begin{description}
\item[Checking Questions/Things to look for] 
\end{description}

\begin{itemize}
\item Does it make sense for the time to be greater or less with more filters? How do you know?
\subitem \textbf{Main Point:} Cross-sectional area has not changed, but total mass has increased, so you would expect to reach terminal velocity faster (time would be shorter)
\item What would you expect ``$n$'' to be?
\end{itemize}

\begin{description}
\item[Grading] 
\end{description}

\begin{itemize}
\item All data is present and reasonable
\item Graph is reasonable (proper axes and use of power curve)
\item Clear statement of the values of ``$n$'' and the drag coefficient $C$
\end{itemize}

\end{flushleft}
\end{noindent}
\end{document}