\documentclass[12pt]{article}

\newcommand{\squishlist}{
   \begin{list}{$\bullet$}
    { \setlength{\itemsep}{0pt}      \setlength{\parsep}{3pt}
      \setlength{\topsep}{3pt}       \setlength{\partopsep}{0pt}
      \setlength{\leftmargin}{1.5em} \setlength{\labelwidth}{1em}
      \setlength{\labelsep}{0.5em} } }

\newcommand{\squishend}{
    \end{list}  }

\usepackage{times}
\usepackage{mathptm}
\usepackage{fullpage}

\begin{document}

\begin{center}
{\bf{Regis University -- Physics 305A -- Fall 2017}} \\
{\bf{Lab 12: Video Analysis}} \\
\end{center}

\noindent {\bf {Tutorial: Shooting a basketball}}

First, let's work through a tutorial that comes with the Logger Pro program.
Start the application and open the file ``12 Video Analysis.cmbl'' in the 
Tutorials folder.  Using the sample video there, follow the instructions to 
set the scale and to track the motion of the basketball from the time it leaves 
the player's hand to the time it hits the floor.
Please analyze this data in two ways:
\squishlist
\item Start with graphs of $x$ vs. $t$ and $y$ versus $t$.  Fit each of
  them to an appropriate functional form (linear or parabolic).  
  From the fit parameters, calculate the initial $x$ and $y$ velocities and 
  the $y$ acceleration.
\item Next, switch to graphs of $v_x$ vs. $t$ and $v_y$ versus $t$.  To what 
  extent is $v_x$ constant?  Fit a linear model to your graph of $v_y$; 
  what is the value of $y$ acceleration?  Does it match what you 
  would expect, and does it match the value that you found from the graph
  of $y$ vs. $t$?
\squishend

\medskip
\noindent {\bf {Your own video}}

Next, choose something to analyze yourself with video analysis.
If you can't think of anything better, you could film a coffee filter (or a
stack of them) dropping and determine how long it takes for it to reach its terminal
speed.  

You can use your own phone to record a video, or we have cameras
that you can use.  Remember that you need to include something in the frame 
to set the scale; it should be in the plane of motion of the object, and it 
might be a measured dimension on the moving object itself.
Import your video into Logger Pro, and analyze it as you did with the
basketball.


\medskip
\noindent {\bf {Other tasks for today}}

\squishlist
\item You should tell me about your ideas for an experimental project.
\item You should continue working with your group on your computational project.
\squishend

\medskip
\noindent {\bf {Computational project presentations}}

\noindent In your group's presentation on Nov. 30, you will need to 
\squishlist
\item introduce the computational problem, 
\item outline the structure of your program, 
\item demonstrate it (it should produce an animation that will be interesting 
  to watch), and 
\item explain the scientific conclusions that you can draw from experimenting 
  with the program and observing the output.
\squishend
You will need to prepare visual slides to accompany your presentation.

\end{document}
