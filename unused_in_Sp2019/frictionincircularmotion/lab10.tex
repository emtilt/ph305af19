\documentclass[11pt]{article}

\newcommand{\squishlist}{
   \begin{list}{$\bullet$}
    { \setlength{\itemsep}{0pt}      \setlength{\parsep}{3pt}
      \setlength{\topsep}{3pt}       \setlength{\partopsep}{0pt}
      \setlength{\leftmargin}{1.5em} \setlength{\labelwidth}{1em}
      \setlength{\labelsep}{0.5em} } }

\newcommand{\squishend}{
    \end{list}  }

\usepackage{times}
\usepackage{mathptm}
\usepackage{fullpage}
\usepackage{graphicx}

\begin{document}

\begin{center}
{\bf{Regis University -- Physics 305A -- Fall 2017}} \\
{\bf{Lab 10: Friction in Circular Motion}} \\
\end{center}

This lab may feel like a ``mash-up'' of several others that you have done 
recently -- it involves a combination of friction, circular motion, and coins.

You will place a penny on the aluminum plate whose moment of inertia you 
once measured, and you will allow it to spin faster and faster 
on the rotational motion sensor until the penny flies off.  
You should choose a position for the penny, measure the radial position of 
the penny from its center, then let it fly.   You can stop the rotation with 
your hand at that instant.  Just at the moment when the penny starts to move:

\squishlist
\item What is the angular speed of the penny?
\item What is the linear speed of the penny?
\item What is the radial (centripetal) component of the penny's acceleration?
\item What is the azimuthal (tangential) component of the penny's acceleration?
\item What is the magnitude of the penny's acceleration?  Did one component
end up almost completely dominant over the other?
\item What is the coefficient of static friction $\mu_s$ between the penny and the aluminum plate?
\squishend

Record enough data to be able to reliably estimate the uncertainty 
of your measurement of $\mu_s$, starting the penny from the same radial 
position each time.

Finally: choose a very different radial position for the penny, and make a 
prediction of the angular speed at which the penny will start to move from 
there, assuming that $\mu_s$ remains the same.  Then, after you have made
your prediction, do the experiment, and compare.

\end{document}
