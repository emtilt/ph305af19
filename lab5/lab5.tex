\documentclass[10pt]{article}

\newcommand{\squishlist}{
   \begin{list}{$\bullet$}
    { \setlength{\itemsep}{0pt}      \setlength{\parsep}{3pt}
      \setlength{\topsep}{3pt}       \setlength{\partopsep}{0pt}
      \setlength{\leftmargin}{1.5em} \setlength{\labelwidth}{1em}
      \setlength{\labelsep}{0.5em} } }

\newcommand{\squishend}{
    \end{list}  }

\usepackage{times}
\usepackage{mathptm}
\usepackage{fullpage}
\usepackage{url}
\begin{document}

\begin{center}\vspace{-15ex}
{\bf{Regis College Physics 205A, Fall 2019}} \\
{\bf{Lab 5: Conservation of Momentum by Video Analysis}} \\
\end{center}

In this lab, you will experimentally test the ideas that we have discussed
about conservation of momentum in collisions. We will use a set of battery-operated ``hover pucks'' that direct a stream 
of air downward
so that they can move with little friction on a smooth surface.  (The effect
is similar to an air hockey table, but the air source is in the puck
rather than in the table.)  Each group will use a cell phone camera to record video, which you will analyze.

\section*{Important notes to keep in mind throughout the lab:}

\begin{itemize}
\item The pucks use up batteries fast, so be sure to \textbf{only} turn them on during their brief collisions. I have limited extra batteries.
\item Verify that the batteries in your puck are strong enough during every experiment. When turned on and placed on a level surface, the puck should experience almost zero friction from the surface.
\end{itemize}

\section{Tutorial}
This portion of the lab should not be included in your report; it is just to teach the software.
\begin{itemize}
\item Start LoggerPro. Go to ``File'' then ``Open.'' By default, you should be in a folder called ``Experiments.'' Navigate to the subfolder ``Tutorials'' and open  the 
file ``12 Video Analysis.cmbl'' in the Tutorials folder.
\item Complete the tutorial. It may be helpful to resize the video before adding data points. Do not proceed past this step until you understand the basic procedure for video analysis and can obtain a fit coefficient consistent with $-4.9~\rm m/s^2$, as described in the tutorial. 

\end{itemize}


\section{Experiments}



\begin{enumerate}

\item	Determine the mass of each puck. Note that the pucks do not all have equal mass, so be sure you keep track of which is which somehow.
\item	Find a suitable flat surface on which to film your collision. You can use a lab table, but you may need to level it, and you need to be able to film from above. The hallway floor is another strategy. Devise your own strategy.
\item	Ensure that an object of known size is in the shot, so that you can set the video scale in Logger Pro.
\item	Choose a cell phone that will be your camera. I’m unsure what video formats modern Apple phones use, but I know this works with nearly any Android phone. If you have the option, set the resolution of the video fairly low and the framerate fairly high, so that the file size is reasonable.
\item	Prepare to film collisions. One team member should hold the cell phone directly above the point of impact. It is very important that you keep the camera stationary and pointed directly downward at the collision. Start the filming right before the collision, and stop the filming after it. Each video should only be seconds long.
\item	At least two other team members will control the two pucks. Successfully film several collisions such as:
\begin{itemize}
\item	A collision between a moving puck and one that is at rest (one team member will have to hold the stationary puck stationary until just before the collision), but don’t make it exactly head-on 
\item	A glancing collision between two pucks moving in roughly opposite directions
\item	A head to head collision between two pucks
\item     A collision with pucks with velcro on them so that they stick together
\end{itemize}
\item	Ensure you captured the collisions on video successfully. If the camera moved, or was at an angle, or didn’t fully capture the collision, try again until you get a good video. The entire lab depends on having good video data.
\item	Email the videos to yourself and download them to the lab computer, or use another method to get them onto the lab computers.
\item	Choose an interesting video to analyze first, and open LoggerPro. Depending on the video format your phone used, LoggerPro may be able to open it directly via Insert $\rightarrow$ Movie.
\item	If the movie will not open in a working state, you need to convert your movie format. You can use an online converter like \url{https://video.online-convert.com/convert-to-avi}. In the “Select Video Codec” option, set it to “mjpeg”. Convert and download your movie. Open it in Logger Pro via Insert $\rightarrow$ Movie. Be sure it plays as expected.
\item	Use the same techniques as in the tutorial to obtain position and velocity data for both pucks. Remember to set your scale, and be sure to add a data series for the second puck by using the sixth buttom from the top on the right-side video controls.
\item	Copy all of the position and velocity data for the two pucks to Excel or another spreadsheet. You should have \textbf{nine} columns of non-zero data, with precision to at least several decimal places.
\item 	Create new columns for the x and y momenta of puck 1, the x and y momenta of puck 2, and the total x and total y momenta for the system. Use formulae to calculate the values for these cells. You should have added six more columns of data in this step.
\item	Make a scatter plot of the x momentum of puck 1 as a function of time. On the same plot, right click and use “Select Data” to add two more data series to the plot: one that shows the x momentum for puck 2 and one that shows the total x momentum of the system. You should now have three data series plotted on one graph.
\item	Repeat for the y momenta, so that you have an additional graph again with three data series displayed.
\item What do your results say? Do your graphs behave as expected? Based on the standard deviation across several frames of video, estimate the uncertainty of the measurement of each of these momentum components.  Is each component of the total momentum of the system conserved (within your estimated experimental uncertainty)?
\item	You have not yet encountered the formal definition of energy in class, but it is another conserved quantity like momentum. The kinetic energy of an object, which is a scalar quantity, can be thought of as the energy associated with the object's movement; a moving object has energy. You can calculate the kinetic energy of an object via $K=\frac{1}{2}mv^2$ where $v$ is the object's speed. Determine if the total kinetic energy of the two puck system is conserved during the collision. While total energy is a conserved quantity, it doesn't necessarily have to remain in the form of \textit{kinetic} energy. We broadly group collisions into three categories:
\begin{itemize}
\item Elastic collisions conserve total system kinetic energy.
\item Inelastic collisions do not conserve kinetic energy.
\item Super-elastic collisions \textit{gain} kinetic energy during the collision.
\end{itemize}
How would you describe your collision (keeping in mind your experimental uncertainty)?
\item Repeat steps 11 through 19 for as many of your other two-puck videos 
  as time allows.
\end{enumerate}

\end{document}
