\documentclass[11pt]{article}

\newcommand{\squishlist}{
   \begin{list}{$\bullet$}
    { \setlength{\itemsep}{0pt}      \setlength{\parsep}{3pt}
      \setlength{\topsep}{3pt}       \setlength{\partopsep}{0pt}
      \setlength{\leftmargin}{1.5em} \setlength{\labelwidth}{1em}
      \setlength{\labelsep}{0.5em} } }

\newcommand{\squishend}{
    \end{list}  }

\usepackage{times}
\usepackage{mathptm}
\usepackage{fullpage}

\begin{document}

\begin{center}
{\bf{Regis College Physics 205A, Fall 2007}} \\
{\bf{Lab 3: Conservation of Momentum by Video Analysis}} \\
\end{center}

In this lab, you will experimentally test the ideas that we have discussed
about conservation of momentum and energy in collisions, and motion of the
center of mass.  

We will use a set of battery-operated ``hover pucks'' that direct a stream 
of air downward
so that they can move with little friction on a smooth surface.  (The effect
is similar to an air hockey table, but the air source is in the puck
rather than in the table.)  Each group will also use a Canon PowerShot A560 
digital camera, which has an excellent movie mode, offering a choice of 
30 frames per second at 640$\times$480 resolution or 60 frames per second at 
320$\times$240 resolution.

There are no prelab questions this week: your recent hard work with 
conservation of momentum in class should have prepared you for the
analysis in this lab.  However, please do read the lab handout and have
at least a quick look at the documentation for the {\em{Tracker}} program.

\bigskip
\noindent {\bf{Procedural notes}}

\begin{enumerate}
\item Try to level your lab table(s) with appropriate shims so that a puck
placed on it/them gains little momentum from gravitational interactions.
\item Determine the mass of each puck.
\item Film several collisions between pucks on the table.  Include at least:
  \begin{itemize}
  \item a collision between a moving puck and one that is at rest.
  \item a nearly head-to-head collision between two pucks.
  \item a more glancing collision between two pucks.
  \item together with another group, a more complicated process involving
    three or four pucks that includes multiple collisions.
  \end{itemize}
Hold the camera as steady as you can, with its sensor plane nearly parallel 
to the plane of the table.  Include an object of known length (such as a ruler)
in each clip.  Check that your video looks reasonable when played back on the 
camera's display.  I recommend that you take at least some of your movies 
twice, once in the high frame rate mode and once in the standard frame 
rate mode.
\item Once all groups have collected their film clips, we will 
migrate {\em{en masse}} to Carroll Hall 240A to import and analyze them 
on the computer.
\item Using the {\em Tracker} software package, select the 
  coordinates of the centers of the pucks in each frame of one of your 
  two-puck collision videos.
\item Determine the $x$ and $y$ components of the momentum vectors of
  each of the pucks before and after the collision.  Based on the standard 
  deviation across several frames, estimate the uncertainty of the measurement 
  of each of these momentum components.  Is each component of the total  
  momentum of the system conserved (within your estimated experimental
  uncertainty)?
\item Compute the kinetic energies of each of the pucks, before and after
  the collision, including uncertainties.  Does the total energy of 
  the system appear to be conserved?
\item Use the {\em Tracker} software's built-in center-of-mass feature to 
  show the trajectory of the center of mass of the two-puck system. 
  Qualitatively describe the motion of the center of mass.  Does it behave
  as you expect?
\item Next, move to the complicated multi-puck interaction video.  Mark 
  the coordinates of the centers of the pucks
  in each frame of this video as well.  In this case, you need not spend
  the time to examine each collision in detail; just look at the motion of
  the center of mass.  Even in this complicated system, does it behave as 
  expected?
\item Repeat steps 5 through 8 for as many of your other two-puck videos 
  as time allows.
\end{enumerate}

\bigskip
\noindent {\bf{Technical hints}}

To set the camera for movie mode, turn the thumbwheel so that the picture of
the movie camera is aligned with the line on the camera.  Turn the camera on
by pressing the ON/OFF button.  To choose between the standard and fast frame
rates, press the FUNC/SET button on the back of the camera (the center of 
the circular control ring), then press on the left or right sides of the 
control ring to select an option, then press FUNC/SET again.  When you're ready 
to start filming, press the shutter release button; press it again to end the 
movie.  To play back your movie on the LCD screen, first press the play/record 
button (an arrow pointing right and a red camera icon, above and left from the 
control ring), then press FUNC/SET twice to play.  Press the play/record button
again to go back to recording mode.

When you are ready to import the movies onto the computer, connect the USB 
cable to the port on the left side of the camera (as seen from the front, 
under a cover marked ``DC IN DIGITAL A/V OUT'') and to the front of the 
computer.  The computer should recognize the camera and offer to open a 
choice of programs: pick the Microsoft Scanner and Camera Wizard. 
Click ``Next'' a few times, then choose a name for your movie collection; a
subfolder within your ``My Pictures'' folder will be created by this name
to hold your movie files.

To start the {\em{Tracker}} program, choose ``Regis College $\rightarrow$
Physics $\rightarrow$ Tracker'' from the Start menu.  Documentation for
this program is available from 
{\em{http://www.cabrillo.edu/$\sim$dbrown/tracker/help/frameset.html}}, which
you should read before starting.

\end{document}
