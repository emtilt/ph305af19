\documentclass[11pt]{article}

\newcommand{\squishlist}{
   \begin{list}{$\bullet$}
    { \setlength{\itemsep}{0pt}      \setlength{\parsep}{3pt}
      \setlength{\topsep}{3pt}       \setlength{\partopsep}{0pt}
      \setlength{\leftmargin}{1.5em} \setlength{\labelwidth}{1em}
      \setlength{\labelsep}{0.5em} } }

\newcommand{\squishend}{
    \end{list}  }

\usepackage{times}
\usepackage{mathptm}
\usepackage{fullpage}
\usepackage{hyperref}

\begin{document}

\begin{center}
{\bf{Regis University -- Physics 305A -- Fall 2019}} \\
{\bf{Lab 1: Computational Tools}} \\
\medskip
\end{center}
\begin{center}
{\bf{Part A -- Spreadsheet Basics}} 
\end{center}

\medskip

\begin{center}
{\bf{Part B -- Computational Physics Activity 1: The Basics of A Python Computer Program}} 
\end{center}

\medskip 

\begin{center}
{\bf{Part C -- Computational Physics Activity 2: Simulating motion at constant velocity}} 

{\em{\tiny Based on activities VP01 and VP02 in the instructor resources that
 accompany Ruth Chabay and Bruce Sherwood's {\em Matter and Interactions} 
 textbook.}}
\end{center}

So far we've talked about tools for doing basic mathematical calculations and analyzing experimental data, but computers have an additional use in physics: solving physics problems by simulating them numerically. In this course, we will use a Python module called VPython which is designed to make simulating physics with 3D graphics very easy to program.

Unfortunately, the Python environment on Colab doesn't support the 3D graphics of VPython, so we'll have to use a different environment (no big deal -- that's the point of different Python environments, to let you use the tools appropriate to the problem you want to tackle). The easiest way to run VPython for the purposes of this lab is in a browser at {\url http://www.glowscript.org/}.\footnote{It is also possible to run VPython locally on a computer without using the internet. On the lab laptops, you can do so by running the VIdle environment, found in the Start menu under Physics, but because they have an older version installed, you must import from \texttt{visual} rather than \texttt{vpython}. If you're feeling \textit{really} adventurous and would like to install VPython on your own laptop, it is a slightly complicated process, with several options described at \url{https://vpython.org/}.} However, there is one difference when running VPython on Glowscript compared to using more ordinary Python environments: the first line of the program must tell Glowscript what kind of code you'll be running, so in our case the first line should be \texttt{GlowScript 2.8 VPython}. When you create a new program on Glowscript, this line should be automatically added.

Please start by watching these three short videos (between 3 and 4 minutes 
  each):
\squishlist
\item {\url http://vpython.org/video01.html}
\item {\url http://vpython.org/video02.html}
\item {\url http://vpython.org/video03.html}
\squishend
Each video ends with a ``challenge'' to you.  Attempt each of these challenges before starting into the next video -- don't move on until you get it working!  

\medskip

\noindent Then, please read this program:
{\small{
\begin{verbatim}
from visual import *

offset = vector(0,-3,0)
position_now = vector(0,0,0)
n = 0
while n < 5:
   sphere(pos=position_now, color=color.red)
   position_now = position_now + offset
   n = n + 1
\end{verbatim}
}}
\noindent Make a prediction of what the program will do, and write down that prediction 
before you run it.  Did it do what you expected?  If not, do you understand
why it did something different?


\noindent Next, please do the same with the following program:
{\small{
\begin{verbatim}
from visual import *

ball = sphere(pos=vector(-5,0,0), radius=0.5, color=color.green)
block = box(pos=vector(-8,0,0), color=color.yellow)
velocity = vector(0.4, 0.6, 0)
delta_t = 0.1
t = 0

while t < 12:
  rate(100)
  ball.pos = ball.pos + velocity * delta_t
  t = t + delta_t
\end{verbatim}
}}
\noindent Again, make a prediction, check it by running the program, and explain
any discrepancy.

Finally, here is the beginning of a program, which you can add on to.
Currently, it draws a magenta sphere just in front of a green wall.
The sphere (named ``particle'') starts at $\vec{r}_i = [-10, 0, 0]$~m.  
A velocity vector is defined so that $\vec{v} = [0.5, 0, 0]$~m/s.
Make a prediction; how long should it take for the sphere to go to
$\vec{r}_f = [10, 0, 0]$~m?

{\small{
\begin{verbatim}
from visual import *

wall = box(pos=vector(0,4,-0.5), length=20, height=10, width=0.1, color=color.green)
particle = sphere(pos=vector(-10,0,0), radius = 0.5, color=color.magenta)
velocity = vector(0.5,0,0)
delta_t = 0.1
t = 0
\end{verbatim}
}}
Starting with this program, add a {\texttt {while}} loop that moves the ``particle''  
along the wall, from the left side to the right side.  Have it stop when it 
gets to the right side; you can refer to the $x$ component of the vector that 
represents the ball's position as {\texttt {particle.pos.x}} in the loop's 
condition expression. 

Have your program print out the elapsed time after the completion of the
loop.  How long did it take?  Does this match your prediction?

So, congratulations!  You have just written a computer program to solve
a physics problem by simulating it numerically.  This is a powerful technique 
that is frequently used by scientists to model systems that are too 
complicated to understand with pencil-and-paper calculations alone.

\end{document}

