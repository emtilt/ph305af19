\documentclass[11pt]{article}

\title{Regis University Physics 305A: General Physics with Calculus Laboratory I, Fall 2019}
\author{Frederick Gray}

\usepackage{url}
\usepackage[cm]{fullpage}
\usepackage{mathptmx}
\usepackage{graphicx}
\setcounter{secnumdepth}{0}

\newcommand{\squishlist}{
   \begin{list}{$\bullet$}
    { \setlength{\itemsep}{0pt}      \setlength{\parsep}{3pt}
      \setlength{\topsep}{3pt}       \setlength{\partopsep}{0pt}
      \setlength{\leftmargin}{1.5em} \setlength{\labelwidth}{1em}
      \setlength{\labelsep}{0.5em} } }

\newcommand{\squishlistB}{
   \begin{list}{--}
    { \setlength{\itemsep}{0pt}      \setlength{\parsep}{3pt}
      \setlength{\topsep}{3pt}       \setlength{\partopsep}{0pt}
      \setlength{\leftmargin}{1.5em} \setlength{\labelwidth}{1em}
      \setlength{\labelsep}{0.5em} } }

\newcommand{\squishend}{
    \end{list}  }

\begin{document}

\begin{center}
\noindent\includegraphics[height=1.8cm] {RegisU_Logo.png}\\
\vspace{-0.5cm}
{\bf{General Physics with Calculus Laboratory I}} \\ 
{\bf{PH 305A (1 semester hour; Fall 2019)}} \\
{{Section RU01: Thursday, 8:10 - 10:50 a.m., Pomponio Family Science Center, room 130}} \\
{{Section RU02: Thursday, 11:10 a.m.-1:50 p.m., Pomponio Family Science Center, room 130}} 
\end{center}

\section{Overview}

The primary goal for this course is for you to build conceptual intuition 
about force, momentum, and energy, the ideas that physicists use to 
describe motion.  In addition, you should learn to solve problems by 
designing and carrying out relevant experiments -- in other words, to put the 
scientific method into practice.  You will also start to learn how to 
build computational models that you can use to understand more complex
physical systems.  You will document the results of your experiments in 
writing, and you will give an oral presentation of the results of a
computational modeling project.

\section{Course Description} 

Complements PH 304A, providing practical, hands-on experience primarily with experiments related to mechanics.

\section{Pre/Co-Requisites}

Co-requisite: PH 304A (General Physics with Calculus I), which in turn lists MT 360A (Calculus I) as a co-requisite.

\section{Instructor}
\squishlist
\item Evan Tilton, Ph.D. (Feel free to call me Evan, Prof. Tilton, Dr. Evan, Hey-You... whatever you're comfortable with. I use he/him/his pronouns.)
\item Office: Carroll 108C
\item Office hours: Monday 1pm-3pm; Tuesday 1pm-2pm; Wednesday 1pm-2pm. But I have an open door policy. You're welcome to talk to me any time I'm here. Outside of scheduled office hours, I might be busy, but I'll chat with you if at all possible. If you want to be sure that I have time to talk, please email me to make sure I'm available.
\item  Email: \url{etilton@regis.edu}.  This is generally the best way to contact me.
\item Phone: 3034584166 (unreliable)
\squishend

\section{Student Learning Outcomes}

In this lab, you should learn to:
\squishlist
\item Describe pheonmena you observe in the real world in terms of physics concepts.
\item Solve problems by designing and carrying out relevant experiments, putting the scientific method into practice.
\item Model and visualize the behavior of physical systems by writing computational simulations.
\item Apply basic data analysis and statistics skills.
\item Communicate the results of experiments and simulations.
\squishend
This course is aligned with the following departmental learning outcomes:
\squishlist
\item Knowledge of the fundamental principles of analytical mechanics, special relativity, electricity and magnetism, quantum mechanics, and statistical mechanics.
\item Ability to apply the principles of physics to solve qualitative and quantitative problems using both analytical and computational methods.
\item Ability to design and conduct experiments.
\item Ability to communicate effectively, both orally and in writing.
\squishend
...and is mapped to these University-level institutional learning outcomes:
\squishlist
\item Knowledge of a discipline or content area.
\item Knowledge of arts, sciences, and humanities.
\item Ability to think critically.
\item Ability to communicate effectively.
\item Ability to use contemporary technology.
\squishend

\section{Presence and absence}

In general, you are expected to attend every lab, arriving on time and prepared to stay for the entire scheduled period.
If you are unavoidably absent because of illness, a significant family emergency, or participation in University-sponsored
events such as intercollegiate athletics, you will be able to negotiate a mutually convenient time with the instructor to make up 
the lab.  Make-up labs in other circumstances are entirely at the discretion of the instructor.  If you were not present to
complete a lab experiment, you will not have the data needed to complete the weekly report.

\section{Weekly reports}

I will ask you to write a {\em short} report on each week's experiment and 
submit it to a dropbox on \url{{http://worldclass.regis.edu}},
normally by the end of the day on the following Monday.
These weekly reports should always contain the following elements:
\squishlist
\item Summary: explain what you did in the lab in one compact paragraph.
\item Data and results: include tables and graphs that show everything 
  that you directly measured and all important quantities that you calculated,
  accompanied by enough written text to explain the meaning of each of the 
  values.  You should always state the units for each number, and each axis 
  of a graph should be labeled with a description of the quantity plotted on 
  it as well as its units.
\item Conclusions: what are your data and results telling you about the world?
  ``I learned a lot about electrons'' is not an appropriate scientific conclusion.  
  Instead, write something more like ``The charge of the electron was measured to be $1.48 \times 10^{-19}$~C, which agrees to within 7.5\% with the accepted value of $1.60 \times 10^{-19}$~C.''
\squishend

Each experiment will also identify specific questions for you to consider.  
If a sentence in the lab handout ends with a question mark, it's a question!
Please make sure that you include your answers to these questions in your 
report, and that you write them in a way that would be understood by a reader 
who only has your report and not the lab handout with the questions.
Do not simply write the answers without context.

Please note that, in these weekly reports, I am generally {\em not} asking you 
to document all of the details of the experimental procedure, as you may
have done in other laboratory classes.  Instead, I hope that you will spend 
your time thinking carefully about the purpose of the experiment and the 
interpretation of the data that you collected.

\section{Computational activities}

Computational models are often used to solve problems that are too
complicated to study with pencil-and-paper analytical calculations.
As the problems at the forefront of physics have become more complex,
and as computers have become more powerful, computational techniques have
become more important.

There will be three in-class computational activities that will allow you
to learn simple programming and computational modeling techniques in the 
VPython environment.  These activities are derived from tutorials distributed
by Ruth Chabay and Bruce Sherwood, authors of the {\em{Matter and Interactions}}
textbook.  You will only have to document completion of these activities
by uploading the final program to a dropbox on \url{{http://worldclass.regis.edu}}.

I will then assign your group a computational modeling problem; each 
group in a section will work on something different.  These problems 
will ask you to write a VPython program that simulates a physical system,
and to use the program to determine the answer to a scientific question.
You will then present your results to the class in a conference-style
session planned for November 14.
In your presentation, you will need to introduce the problem, outline the
structure of your program, demonstrate it (it will produce
an animation that will be interesting to watch), and explain the conclusions
that you can draw from experimenting with the program and observing the output.
You will need to prepare visual slides to accompany your presentation.

\section{Experimental Project}

In the first part of the course, you will become familiar with the equipment
that is available in the lab and with some data collection and analysis
techniques.  At the end, I will ask you to design an experiment that uses
that equipment and those skills to address a physics question that 
interests you.  By November 14, you should have a good idea of your topic
and the general outline of the experimental procedure, and I will ask you to 
put those ideas into writing by that day.  You can then use the last two lab 
periods as work time to refine the procedure and to collect and analyze
your data.  

For this independent project, you will produce a final laboratory report 
that will be much more complete and more formal than your weekly reports.
The format of this report will be modeled on a published journal article, 
so it will include an abstract, an introduction that explains the scientific 
context, a detailed description of the procedure, data and results that are 
distilled into appropriate graphs and tables, and a scientific conclusion.
I will plan to meet with each of you to review a draft of the report before
you submit a revised version by the end of the final examination week.

\section{Grades}
\noindent The work in the course will be weighted as follows to determine
your final grade:
\squishlist
\item Weekly laboratory reports: 50\%
\item Computational tutorial activities: 15\%
\item Computational project: 15\%
\item Experimental project: 20\%
\squishend

Each of these assignments will be evaluated using the following rubrics:
\begin{center}
\begin{tabular}{|l|l|l|l|}
\hline
           & Criteria                                           & Score & Letter \\
\hline
\hline
Check-plus  & Lab complete, analysis essentially correct,  & 100\% & A \\
           & valid conclusions drawn from data,                 &    & \\
           & presentation polished.                             &    & \\
\hline
Check       & Lab complete, only small mistakes in analysis,     & 85\%  & A-/B+ \\
           & valid conclusions drawn from data,         &    & \\
           & presentation adequate.                             &    & \\
\hline
Check-minus & Small parts of lab incomplete or missing,          & 60\%  & C \\
           & mistakes in analysis that lead to substantial & & \\
           & distortions, invalid conclusions drawn from data,  &    & \\
           & or inadequate presentation.               &    & \\
\hline
F           & Substantial parts of lab incomplete or missing.    & 30\%  & F \\
\hline
0           & Absent from lab, or report not submitted.        & 0\%  & F \\
\hline
\end{tabular}
\end{center}
\noindent In borderline cases, I may choose a grade that is halfway between two of
these points, such as ``Check/Check-plus.''  The percentage scores correspond
to a 15-point grade scale:
\begin{center}
\begin{tabular}{ccc}
         & 89-100 A & 85-88 A- \\
81-84 B+ & 74-80 B & 70-73 B- \\
66-69 C+ & 59-65 C & 55-58 C- \\
         & 40-54 D &  \\
         & 0-39 F & \\
\end{tabular}
\end{center}


\section{Collaboration expected}
You are expected to work as part of a laboratory group to perform the 
experiments and gather your data.  This data should be shared with all
members of the group.  You may collaborate with your group to produce 
graphs or other figures that represent your data, which may appear in each of 
your reports.  You are also expected to work together to develop the programs
that implement your computational models.
However, each member of the group must write the text of the 
report independently. 

\section{Academic Honor Code}
All members of the Regis University community exhibit the qualities of honesty, loyalty and trustworthiness in all academic activities, holding themselves and each other accountable for the integrity of the learning community. Regis University students are committed to the highest standards of academic integrity and assume full and complete responsibility for maintaining those standards in the academic environment.

\section{Academic Integrity Violations}

Violations of academic integrity are taken very seriously and include cheating, plagiarism, fabrication, collusion and other forms of academic misconduct.  All violations will be reported with appropriate sanctions applied.  Sanctions can include, but are not limited to failure of an assignment, failure of a course, removal of academic honors, or review of the Academic Integrity Tutorial.  For more serious violations, program suspension, College dismissal or University expulsion may be imposed.  Refer to the Regis College Office of the Academic Dean for further information.  This Academic Honor Code applies to any student enrolled in a course at Regis University or one of its university partners, regardless of the student's home college or program, and will be enforced according to the policies and procedures outlined in the University Academic Integrity Policy.  For the full policy, please see \url{https://www.regis.edu/About-Regis-University/Policies-and-Procedures/Academic-Integrity-Policies.aspx}.

It is the responsibility of each student to review all aspects of the course syllabus and agree to adhere to the Academic Honor Code.  In doing so, the student acknowledges that the work represented in all examinations and other assignments is his or her own and that he or she has neither given nor received unauthorized information.  Furthermore, the student agrees not to divulge the contents of any examination or assignment to another student in this or ensuing semesters.  


I hope that there will be no need to respond to any academic integrity
violations in this course.  However, any response would be consistent with the
classifications of Level I, II, and III violations in the Regis University
Academic Integrity Policy.  In general, the typical penalty for misconduct
such as plagiarism in a laboratory report would be a grade of 0 on that report.
Egregious violations such as deliberate fabrication or falsification of data 
could lead to an F in the course.

\section{Key dates}
The add/drop deadline is September 3. The withdrawal deadline is November 10. Midterm grades are due October 17.
{\small
\section{Counseling}

During the semester, if you find that life stressors are interfering with your academic or personal success, consider contacting the Office of Counseling and Personal Development (OCPD). All full-time Regis College students are eligible for counseling services at no charge. OCPD is located in the Life Direction Center, Room 114 and can be contacted by phone 24/7 at 303-458-3507. For more information, see {\url{http://www.regis.edu/ocpd}} .

\section{Accessibility}

Regis is committed to creating a learning environment that is equitable, inclusive and welcoming. If you have a disability (or think you may have a disability) that may affect your work in this class and feel you need accommodations, contact Student Disability Services and University Testing (SDS/UT) to schedule an appointment and initiate a conversation about reasonable accommodations. To receive any academic accommodation, you must be registered with SDS/UT, which works with students and faculty to identify reasonable accommodations.  SDS/UT  can be reached in Clarke Hall, suite 241, by phone at (303)458-4941,  or by email at \url{disability@regis.edu}. For more information, please visit the SDS/UT's website (\url{https://www.regis.edu/Academics/SDS-UT/Disability-Services/New-Students.aspx}) 
at \url{http://www.regis.edu/disability}. 


\section{Diversity and Inclusion}
At Regis University the term ``diversity'' affirms our Jesuit commitment to build a community of excellence that values inclusion, dignity and the contributions of all our members. 

We strive to:
\squishlist
\item Shape a learning environment characterized by the Jesuit traditions of mutual respect and the pursuit of social justice.
\item Contribute to the richness and vitality of our global Regis community by recognizing our various identities and experiences including, but not limited to, age, gender, race/ethnicity, class, disability, sexual orientation, religion and other forms of human difference.
\item Fulfill our Jesuit Catholic mission through each member of our community by maintaining a humane atmosphere where the human rights of every individual are recognized and respected through words and actions.
\squishend
Should an individual ever feel as though these values are not being upheld in the academic or residential environment, we encourage that person to contact the Office of Diversity, Equity and Inclusive Excellence, LDC 124, \url{diverse@regis.edu}, 303-458-5301. 

\section{Title IX and Regis's Nondiscrimination and Sexual Misconduct Policy}

In the event that you choose to write or speak about having survived sexualized violence, including rape, sexual assault, dating violence, domestic violence, or stalking {\em and specify that this violence occurred while you were a Regis student}, federal and state education laws require that instructors notify the Regis University Title IX Coordinator, Michelle Spradling. She will contact you to let you know about accommodations and support services at Regis and requirements for holding accountable the person who harmed you. To learn more about Title IX, explicitly at Regis, go to \url{http://www.regis.edu/About-Regis-University/University-Offices-and-Services/Campus-Safety/Title-IX.aspx} .

If you do not want the Title IX Coordinator notified, instead of disclosing this information to your instructor, you can speak {\bf confidentially} with counselors in the Office of Counseling and Personal Development (see below) and the Blue Bench, a community agency that focuses on sexual assault and sponsors a 24/7 hotline. They can connect you with support services and discuss options for holding the perpetrator accountable.  The number is 303-322-7273.


\section{Course schedule}
I'm not going to try to give a detailed schedule this year because most of our labs are designed to use internet-connected computers, and they don't work right now. We'll have to adapt on the fly depending on what technology we can get working.
}

\end{document}
